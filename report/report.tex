\documentclass[a4paper]{acmtrans2m}

\usepackage{url}

\markboth{CS497}{Some Recent Advances in Haskell}

\title{CS497 -- Some Recent Advances in Haskell}
\author{Siddharth Agarwal | Y7429\\
  Advisor: Dr. Amey Karkare}
\date{\today}

\begin{abstract}
We look at two recently developed software engineering tools currently available
in Haskell: software transactional memory (along with an implementation in an
imperative language for comparison), and Haskell's nested data parallel library,
currently in development. We pay special attention to how Haskell's unique
features --- in particular, the static separation of functions with side effects
from functions without --- help in making these implementations easier and
faster.
\end{abstract}

% A dummy category to please the cls file
\category{Foo}{Bar}{Baz}

\begin{document}

\maketitle

\section{Introduction}
The computing world today is multi-core. Processor speeds have reached their
limit, so the only way they can become faster now is by including multiple
processing units on one chip. The upshot of this is that the problem of
concurrent programming is now more relevant than ever.

We look at the problem from two different angles:
\begin{itemize}
\item Section \ref{sec:stm} will look at solving problems where threads have
  significant amounts of shared state that is concurrently accessed and mutated.
\item Section \ref{sec:dph} will look at cases where the algorithm does not
  require much state to be shared among threads, but otherwise accesses data in
  non-trivial ways.
\end{itemize}

\section{Software transactional memory}
\label{sec:stm}

\subsection{Motivation}

Concurrent programming brings several problems along with it, of which the
biggest one is the synchronization of access to \textit{shared mutable
  state}. The dominant form of synchronization today is using locks.

Locks are structures with the invariant that only one thread can hold one at
a given instant. They are a powerful tool but can be very difficult to program
with. Programmers must ensure that operations do not conflict, do not deadlock,
and that the locking is of the right granularity \cite{Duffy:2010}. In addition,
locks are not composable: in general, it is not possible to combine thread-safe
sections of code into a larger thread-safe section of code \cite{Harris:2005}.

\subsection{Transactional memory systems}
To combat these issues, several authors have proposed memory systems where the
fundamental unit of synchronization is the \textit{transaction}
\cite{Harris:2007}. Transactions, as a concept borrowed from databases, are
sections of code in which the load and store operations appear to be atomic (all
happen at once). The earliest transactional memory systems
(e.g. \cite{Herlihy:1993}) all required specialized hardware support, but
\textit{software transactional memory (STM)}, which required no hardware support
beyond the \textit{load-linked/store-conditional} construct \cite{Shavit:1995}, or
equivalently the \textit{compare-and-swap} operation \cite{Harris:2003}, quickly
arose.

\subsection{Language support}

Later, \citeN{Harris:2003} described a system based on critical conditional
regions (CCR) \cite{Hoare:1972} where transaction support was integrated into
the Java language syntax. Such systems, while quite successful, still suffered
from a few problems \cite{Harris:2005,Duffy:2010}:

\begin{itemize}
\item \textbf{Overhead.} To avoid A-B-A problems, each location of transactional
  memory needs a version number. Since these systems proposed to make all memory
  transactional, a naive implementation would mean twice the memory usage. A few
  solutions were proposed, including having a common version number for a group
  of words instead of a separate version number for each word
  \cite{Harris:2003}.
\item \textbf{I/O.} I/O operations are generally irreversible, which causes
  problems with transactions (since they might need to roll back). For
  output-only operations, one might be able to buffer them until one knows the
  transaction is successful, but that wouldn't work for operations which require
  input. A different solution is to forbid any I/O operations, but languages
  like Java do not have any static information about which functions have I/O,
  so they must detect such cases and abort at run-time.
\item \textbf{Weak vs. strong atomicity.} These systems did not restrict the
  same memory location from being accessed both inside and outside transactional
  sections. One could either guarantee \textit{strong} transactional semantics
  outside transactional blocks too (which is equivalent to wrapping every
  non-transactional statement in a one-line transactional block and leads to
  poor performance), or be \textit{weakly} atomic and make no such
  guarantees. One could also avert the problem by making the mutable memory
  accessible inside transactional blocks \textit{disjoint} from mutable memory
  accessible outside them.
\end{itemize}

\subsection{STM in Haskell}

Haskell, as a statically-typed pure functional language with limited and
controlled mutable state, is uniquely suited to software transactions
\cite{Harris:2005}. In Haskell, any mutation of state has to be done inside a
function whose type contains the \texttt{IO} monad. What \citeN{Harris:2005}
propose is a monad for STM operations, called \texttt{STM}.

\begin{itemize}
\item To minimize \textbf{overhead}, they require that any transactional memory
  explicitly be declared as \texttt{TVar}. This means that only a very small
  number of memory locations are transactional, and that each location can contain
  a separate version number while still not adding much overhead.
\item To solve the \textbf{I/O} problem, they forbid general I/O operations inside
  any function marked \texttt{STM}. In Haskell, all I/O operations are marked with
  the \texttt{IO} monad, and since Haskell is statically-typed, the compiler knows
  about these operations. This means that I/O can be statically forbidden inside
  \texttt{STM} sections.
\item To avert the \textbf{weak vs. strong atomicity} problem, they do not allow
  transactional memory to be accessed outside \texttt{STM} blocks. All
  transactional memory must be accessed using the pair of functions
  \texttt{readTVar} and \texttt{writeTVar}, and since these functions are marked
  as \texttt{STM}, they can only be called inside \texttt{STM} blocks.
\end{itemize}

\subsection{Current status and future directions}

Haskell's STM system currently ships with releases of the Glasgow Haskell compiler.
It is perhaps the only STM to now be mature enough to be used in production systems
\cite{Stewart:2009}. However, work still remains to be done in integrating memory
transactions with other sorts of transactional providers, such as databases and
transactional file systems \cite{SPJ:2006}.

\section{Nested data parallelism}
\label{sec:dph}

A different approach to concurrent programming, suitable for certain kinds of
applications, is via \textit{data parallelism}. Data parallelism is
characterized by large sets of data on which transformations need to be done,
and by the fact that these transformations can be done on individual data
units mostly independently. Such parallelism can be used in a wide variety of
places, from calculating the dot product of a pair of vectors to blending
images.

However, traditionally data parallelism libraries require the data to be stored
as flat arrays. This is fine for simple applications like blending images, but
can be quite inconvenient to the programmer for anything more complicated. One
possible way to help the programmer would be to allow nested data structures to
be provided, transform them into flat structures, operate on them and finally
convert them back to nested structures; indeed such a solution was proposed in
the early '90s \cite{Blelloch:1993,Blelloch:1996}. However, this solution required
the programmer to program in a special language called NESL, and was limited in
other ways \cite{Chakravarty:2007}.

\bibliographystyle{acmtrans}
\bibliography{report}
\end{document}
