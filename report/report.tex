\documentclass[a4paper]{acmtrans2m}

\title{Some Recent Advances in Haskell}
\author{Siddharth Agarwal | Y7429\\
  Advisor: Dr. Amey Karkare}
\date{\today}

\setcounter{tocdepth}{2}

\begin{document}

\maketitle

\tableofcontents

\begin{abstract}
We look at two recently developed software engineering tools currently available
in Haskell: software transactional memory (along with an implementation in an
imperative language for comparison), and Haskell's nested data parallel library,
currently in development. We pay special attention to how Haskell's unique
features --- in particular, the static separation of functions with side effects
from functions without --- help in making these implementations easier and
faster.
\end{abstract}

\section{Software Transactional Memory}

\subsection{Motivation}

The computing world is becoming multi-core. Processor speeds have reached their
limits, so the only way to become faster now is to include multiple processing
units on one chip. The upshot of this is that the problem of concurrent
programming is now more relevant than ever.

Concurrent programming brings several problems along with it, of which the
biggest one is the synchronization of access to \textit{shared mutable
  state}. The dominant form of synchronization today is using locks.

Locks are structures with the invariant that only one thread can hold a lock at
a time. They are a powerful tool but can be very difficult to program
with. Programmers must ensure that operations do not conflict, do not deadlock,
and that the locking is of the right granularity. In addition, locks are not
composable: in general, it is not possible to combine thread-safe sections of
code into a larger thread-safe section of code \cite{Harris:2005}.

\bibliographystyle{acmtrans}
\bibliography{report}
\end{document}
